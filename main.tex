\documentclass[english,brazil,a4paper,utf8]{utfpr-cm-monograph}

\usepackage{latexscholar-cite}
\usepackage{latexscholar-translation}
\usepackage{latexscholar-bibtex}

% Este comando não é necessário: utilizei apenas para deixar o latex2rtf
% feliz (e descobrir a codificação do texto).
\usepackage[utf8]{inputenc}

% Define o caminho das figuras
\graphicspath{{images/}}
\graphicspath{{img/}}


% Configura pacotes para a tabela do cronograma
\usepackage{tabularx}
\usepackage{multirow}
\usepackage{array}
\newcommand{\fullcell}{\multicolumn{1}{>{\columncolor[gray]{0.5}}c}{}}
\newcommand{\fullcellline}{\multicolumn{1}{>{\columncolor[gray]{0.5}}c|}{}}
\newcommand{\mc}[3]{\multicolumn{#1}{#2}{#3}}
\newcommand{\y}{\rule{8pt}{4pt}}
\newcommand{\n}{\hspace*{8pt}} 

% Desativa hifenização, mantendo o texto justificado.
% \tolerance=1
% \emergencystretch=\maxdimen
% \hyphenpenalty=10000
% \hbadness=10000

% Relaxa restrições para quebra de linha (evita linhas que
% ultrapassam a margem)
\sloppy


% Dados que não mudam sobre a UTFPR
\university{Universidade Tecnológica Federal do Paraná}
\universityunit{Departamento Acadêmico de Computação}
\program{Curso de Bacharelado em Ciência da Computação}
\degree{Bacharel}
\degreearea{Ciência da Computação}
\goal{Trabalho de Conclusão de Curso}
\address{Campo Mourão -- PR}


% Dados que devem ser alterados
\author{}
\title{}
\advisor{}
\coadvisor{} % nome do co-orientador do trabalho, caso exista
\depositshortdate{} % ano de entrega da monografia


\begin{document}

\frontmatter
\maketitle

\begin{resumo}
%Fazer assim:\\
%Contexto / Problema: escrever 3 ou 4 frases;\\
%Solução Proposta: escrever 2 ou 3 frases;\\
%Resultados esperados: escrever 2 ou 3 frases;\\
\end{resumo}


\tableofcontents
\listoffigures
\listoftables


\mainmatter
\chapter{Introdução}


% Motivação e justificativa

% Objetivos e metas

% Organização do texto
% Capítulo com embasamento teórico e trabalhos relacionados (estado da arte)
%
\chapter{Referencial teórico}
\label{chapter:background}

% Antes de começar a seção, descreva o propósito deste capítulo, evidenciando
% a importância dele para a compreensão da monografia
%
Este capítulo apresenta um


\section{...}


% Arremate do capítulo, sintetizando resultados, com observações críticas
% que associam o texto deste capítulo com o próximo.
%
\section{Considerações finais}


\chapter{Proposta}
\label{chapter:proposta}

% Antes de começar a seção, descreva o propósito deste capítulo, evidenciando
% a importância dele para a compreensão da monografia
%

\section{Método}




\section{Cronograma}
\label{sec:cronograma}

As atividades descritas na \cref{tab:Cronograma} serão executadas da seguinte maneira:

\begin{itemize}
	\item \textbf{Revisão Bibliográfica:} \ldots
	\\\textbf{Resultado esperado:} \ldots
	
	\item \textbf{Escrita da Monografia:} \ldots
	\\\textbf{Resultado esperado:} \ldots
	
	\item \textbf{Defesa:} \ldots
	\\\textbf{Resultado esperado:} \ldots
\end{itemize}

\begin{table}[h]
	\footnotesize
	\caption{Cronograma de atividades.}
	\label{tab:Cronograma}
	\setlength{\tabcolsep}{0pt}
	\centering
	\begin{tabular}{|p{9cm}|c|c|c|c|c|c|c|c|c|c|c|c|}
		\hline
		\textbf{Atividade} 	&\ Jan. &\ Fev. &\ Mar. &\ Abr. &\ Mai. &\ Jun. &\ Jul. \\\hline
		Revisão bibliográfica       & \y    & \y    &\y     &\y      &\y      &\y      &\n    \\\hline
		Escrita da monografia       & \n    & \n    & \y    & \y     & \y     & \y     & \n   \\\hline
		Defesa da monografia        & \n    & \n    & \n    & \n     & \n     & \n     & \y   \\\hline
	\end{tabular}
\end{table}



\bibliographystyle{abnt-alf}
\bibliography{main} % geração automática das referências a partir do arquivo main.bib

\backmatter

\end{document}
