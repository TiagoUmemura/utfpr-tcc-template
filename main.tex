\documentclass[12pt,english,brazil,a4paper,utf8,oneside]{utfpr-tcc}

% Este comando não é necessário: utilizei apenas para deixar o latex2rtf
% feliz (e descobrir a codificação do texto).
\usepackage[utf8]{inputenc}

% Suporte a figuras e subfiguras
\usepackage{graphics}
\usepackage{subfigure}

% Suporte a tabelas (principalmente do cronograma)
\usepackage{tabularx}
\usepackage{multirow}
\usepackage{array}
\usepackage{tabularx}
\usepackage{colortbl}
\usepackage{hhline}
\usepackage{xcolor}

% Elementos geralmente utilizados na tabela do cronograma
\newcommand{\fullcell}{\multicolumn{1}{>{\columncolor[gray]{0.5}}c}{}}
\newcommand{\fullcellline}{\multicolumn{1}{>{\columncolor[gray]{0.5}}c|}{}}
\newcommand{\mc}[3]{\multicolumn{#1}{#2}{#3}}
\newcommand{\y}{\rule{8pt}{4pt}}
\newcommand{\n}{\hspace*{8pt}} 

% Define o caminho das figuras
\graphicspath{{images/}}

% Dados do curso que não precisam de alteração
\university{Universidade Tecnológica Federal do Paraná}
\universityen{Federal University of Technology -- Paraná}
\universityunit{Departamento Acadêmico de Computação}
\address{Campo Mourão}
\addressen{Campo Mourão, PR, Brazil}
\documenttype{Monografia}
\documenttypeen{Monograph}
\degreetype{Graduação}


%%%%%%%%%%%%%%%%%%%%%%%%%%%%%%%%%%%%%%%%%%%%%%%%%%%%%%%%%%%%%%%%%%%%%%%%%%%%%
% Alterar daqui para baixo
%%%%%%%%%%%%%%%%%%%%%%%%%%%%%%%%%%%%%%%%%%%%%%%%%%%%%%%%%%%%%%%%%%%%%%%%%%%%%

% Dados do curso. Caso seja BCC:
\program{Curso de Bacharelado em Ciência da Computação}
\programen{Undergradute Program in Computer Science}
\degree{Bacharel}
\degreearea{Ciência da Computação}
% Caso seja TSI:
% \program{Curso Superior de Tecnologia em Sistemas para Internet}
% \programen{Undergradute Program in Tecnology for Internet Systems}
% \degree{Tecnólogo}
% \degreearea{Tecnologia em Sistemas para Internet}


% Dados da disciplina. Escolha uma das opções e a descomente:
% TCC1:
\goal{Proposta de Trabalho de Conclusão de Curso de Graduação}
\course{Trabalho de Conclusão de Curso 1}
% TCC2:
% \goal{Trabalho de Conclusão de Curso de graduação}
% \course{Trabalho de Conclusão de Curso 2}


% Dados do TCC (precisa alterar)
\author{}  % Seu nome
\title{} % Título do trabalho
\titleen{} % Título traduzido para inglês
\advisor{} % Nome do orientador. Lembre-se de prefixar com "Prof. Dr.", "Profª. Drª.", "Prof. Me." ou "Profª. Me."}
% \coadvisor{} % Nome do coorientador, caso exista. Caso não exista, comente a linha.
\depositshortdate{2016} % Ano em que depositou este documento

% Dados da ficha catalografica. Ela é opcional, mas é uma boa ideia inserí-la. Exemplos para geração (http://fichacatalografica.sibi.ufrj.br/)
\fichacatautor{}  % Nome conforme citado (ou seja, no formato "Sobrenome, Nome").
\fichacatbib{Biblioteca da UTFPR de Campo Mourão} % Não alterar
\fichacatpum{M488} % Código Cutter-Sanborn. Use a primeira letra do sobrenome seguido do número conforme as primeiras letras do sobrenome e a tabela http://www.amormino.com.br/cutter-sanborn/cutter1.html
\fichacatpalcha{} % Assuntos do trabalho. Cada item deve ser enumerado e separado por ponto: 1. xxx. 2. yyy. 3. zzz.
\fichacatpdois{} % Deixar em branco


\begin{document}
	
\frontmatter
\maketitle

\begin{resumo}
% TODO: se possível, escreva um resumo estruturado. Para TCC 1, o resumo estruturado teria os seguintes elementos:
% \textbf{Contexto:} \\
% \textbf{Objetivo:} \\
% \textbf{Método:} \\
% \textbf{Resultados esperados:} 
% ou, para TCC 2:
% \textbf{Contexto:} \\
% \textbf{Objetivo:} \\
% \textbf{Método:} \\
% \textbf{Resultados:} \\
% \textbf{Conclusões:}

% Palavras-chaves, separadas por ponto (tente não definir mais do que cinco)
\palavraschaves{}
\end{resumo}



% Caso seja TCC 2, precisa traduzir o resumo e as palavras-chaves para inglês:
% \begin{abstract}
% \textbf{Context:}
% \textbf{Objective:}
% \textbf{Method:}
% \textbf{Results:}
% \textbf{Conclusions:}

% Palavras-chaves em inglês, separadas por ponto.
% \keywords{}
% \end{abstract}



% Listas (opcionais, mas recomenda-se a partir de 5 elementos)
\listoffigures
\listoftables

% Sumário
\tableofcontents

\mainmatter
% TODO: incluir arquivos latex com os capítulos
% \chapter{Introdução}


% Motivação e justificativa

% Objetivos e metas

% Organização do texto
% % Capítulo com embasamento teórico e trabalhos relacionados (estado da arte)
%
\chapter{Referencial teórico}
\label{chapter:background}

% Antes de começar a seção, descreva o propósito deste capítulo, evidenciando
% a importância dele para a compreensão da monografia
%
Este capítulo apresenta um


\section{...}


% Arremate do capítulo, sintetizando resultados, com observações críticas
% que associam o texto deste capítulo com o próximo.
%
\section{Considerações finais}


% \chapter{Proposta}
\label{chapter:proposta}

% Antes de começar a seção, descreva o propósito deste capítulo, evidenciando
% a importância dele para a compreensão da monografia
%

\section{Método}




\section{Cronograma}
\label{sec:cronograma}

As atividades descritas na \cref{tab:Cronograma} serão executadas da seguinte maneira:

\begin{itemize}
	\item \textbf{Revisão Bibliográfica:} \ldots
	\\\textbf{Resultado esperado:} \ldots
	
	\item \textbf{Escrita da Monografia:} \ldots
	\\\textbf{Resultado esperado:} \ldots
	
	\item \textbf{Defesa:} \ldots
	\\\textbf{Resultado esperado:} \ldots
\end{itemize}

\begin{table}[h]
	\footnotesize
	\caption{Cronograma de atividades.}
	\label{tab:Cronograma}
	\setlength{\tabcolsep}{0pt}
	\centering
	\begin{tabular}{|p{9cm}|c|c|c|c|c|c|c|c|c|c|c|c|}
		\hline
		\textbf{Atividade} 	&\ Jan. &\ Fev. &\ Mar. &\ Abr. &\ Mai. &\ Jun. &\ Jul. \\\hline
		Revisão bibliográfica       & \y    & \y    &\y     &\y      &\y      &\y      &\n    \\\hline
		Escrita da monografia       & \n    & \n    & \y    & \y     & \y     & \y     & \n   \\\hline
		Defesa da monografia        & \n    & \n    & \n    & \n     & \n     & \n     & \y   \\\hline
	\end{tabular}
\end{table}


% \include{resultados-preliminares}

\bibliographystyle{abnt-alf}
\bibliography{main} % geração automática das referências a partir do arquivo main.bib

\backmatter
\end{document}
